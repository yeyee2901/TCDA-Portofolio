% Options for packages loaded elsewhere
\PassOptionsToPackage{unicode}{hyperref}
\PassOptionsToPackage{hyphens}{url}
%
\documentclass[
]{article}
\usepackage{amsmath,amssymb}
\usepackage{lmodern}
\usepackage{ifxetex,ifluatex}
\ifnum 0\ifxetex 1\fi\ifluatex 1\fi=0 % if pdftex
  \usepackage[T1]{fontenc}
  \usepackage[utf8]{inputenc}
  \usepackage{textcomp} % provide euro and other symbols
\else % if luatex or xetex
  \usepackage{unicode-math}
  \defaultfontfeatures{Scale=MatchLowercase}
  \defaultfontfeatures[\rmfamily]{Ligatures=TeX,Scale=1}
\fi
% Use upquote if available, for straight quotes in verbatim environments
\IfFileExists{upquote.sty}{\usepackage{upquote}}{}
\IfFileExists{microtype.sty}{% use microtype if available
  \usepackage[]{microtype}
  \UseMicrotypeSet[protrusion]{basicmath} % disable protrusion for tt fonts
}{}
\makeatletter
\@ifundefined{KOMAClassName}{% if non-KOMA class
  \IfFileExists{parskip.sty}{%
    \usepackage{parskip}
  }{% else
    \setlength{\parindent}{0pt}
    \setlength{\parskip}{6pt plus 2pt minus 1pt}}
}{% if KOMA class
  \KOMAoptions{parskip=half}}
\makeatother
\usepackage{xcolor}
\IfFileExists{xurl.sty}{\usepackage{xurl}}{} % add URL line breaks if available
\IfFileExists{bookmark.sty}{\usepackage{bookmark}}{\usepackage{hyperref}}
\hypersetup{
  pdftitle={Yu-Gi-Oh! Meets Data Analysis!},
  pdfauthor={Gabriel SH (yeye), tutored by Geeked.id},
  hidelinks,
  pdfcreator={LaTeX via pandoc}}
\urlstyle{same} % disable monospaced font for URLs
\usepackage[margin=1in]{geometry}
\usepackage{color}
\usepackage{fancyvrb}
\newcommand{\VerbBar}{|}
\newcommand{\VERB}{\Verb[commandchars=\\\{\}]}
\DefineVerbatimEnvironment{Highlighting}{Verbatim}{commandchars=\\\{\}}
% Add ',fontsize=\small' for more characters per line
\usepackage{framed}
\definecolor{shadecolor}{RGB}{248,248,248}
\newenvironment{Shaded}{\begin{snugshade}}{\end{snugshade}}
\newcommand{\AlertTok}[1]{\textcolor[rgb]{0.94,0.16,0.16}{#1}}
\newcommand{\AnnotationTok}[1]{\textcolor[rgb]{0.56,0.35,0.01}{\textbf{\textit{#1}}}}
\newcommand{\AttributeTok}[1]{\textcolor[rgb]{0.77,0.63,0.00}{#1}}
\newcommand{\BaseNTok}[1]{\textcolor[rgb]{0.00,0.00,0.81}{#1}}
\newcommand{\BuiltInTok}[1]{#1}
\newcommand{\CharTok}[1]{\textcolor[rgb]{0.31,0.60,0.02}{#1}}
\newcommand{\CommentTok}[1]{\textcolor[rgb]{0.56,0.35,0.01}{\textit{#1}}}
\newcommand{\CommentVarTok}[1]{\textcolor[rgb]{0.56,0.35,0.01}{\textbf{\textit{#1}}}}
\newcommand{\ConstantTok}[1]{\textcolor[rgb]{0.00,0.00,0.00}{#1}}
\newcommand{\ControlFlowTok}[1]{\textcolor[rgb]{0.13,0.29,0.53}{\textbf{#1}}}
\newcommand{\DataTypeTok}[1]{\textcolor[rgb]{0.13,0.29,0.53}{#1}}
\newcommand{\DecValTok}[1]{\textcolor[rgb]{0.00,0.00,0.81}{#1}}
\newcommand{\DocumentationTok}[1]{\textcolor[rgb]{0.56,0.35,0.01}{\textbf{\textit{#1}}}}
\newcommand{\ErrorTok}[1]{\textcolor[rgb]{0.64,0.00,0.00}{\textbf{#1}}}
\newcommand{\ExtensionTok}[1]{#1}
\newcommand{\FloatTok}[1]{\textcolor[rgb]{0.00,0.00,0.81}{#1}}
\newcommand{\FunctionTok}[1]{\textcolor[rgb]{0.00,0.00,0.00}{#1}}
\newcommand{\ImportTok}[1]{#1}
\newcommand{\InformationTok}[1]{\textcolor[rgb]{0.56,0.35,0.01}{\textbf{\textit{#1}}}}
\newcommand{\KeywordTok}[1]{\textcolor[rgb]{0.13,0.29,0.53}{\textbf{#1}}}
\newcommand{\NormalTok}[1]{#1}
\newcommand{\OperatorTok}[1]{\textcolor[rgb]{0.81,0.36,0.00}{\textbf{#1}}}
\newcommand{\OtherTok}[1]{\textcolor[rgb]{0.56,0.35,0.01}{#1}}
\newcommand{\PreprocessorTok}[1]{\textcolor[rgb]{0.56,0.35,0.01}{\textit{#1}}}
\newcommand{\RegionMarkerTok}[1]{#1}
\newcommand{\SpecialCharTok}[1]{\textcolor[rgb]{0.00,0.00,0.00}{#1}}
\newcommand{\SpecialStringTok}[1]{\textcolor[rgb]{0.31,0.60,0.02}{#1}}
\newcommand{\StringTok}[1]{\textcolor[rgb]{0.31,0.60,0.02}{#1}}
\newcommand{\VariableTok}[1]{\textcolor[rgb]{0.00,0.00,0.00}{#1}}
\newcommand{\VerbatimStringTok}[1]{\textcolor[rgb]{0.31,0.60,0.02}{#1}}
\newcommand{\WarningTok}[1]{\textcolor[rgb]{0.56,0.35,0.01}{\textbf{\textit{#1}}}}
\usepackage{graphicx}
\makeatletter
\def\maxwidth{\ifdim\Gin@nat@width>\linewidth\linewidth\else\Gin@nat@width\fi}
\def\maxheight{\ifdim\Gin@nat@height>\textheight\textheight\else\Gin@nat@height\fi}
\makeatother
% Scale images if necessary, so that they will not overflow the page
% margins by default, and it is still possible to overwrite the defaults
% using explicit options in \includegraphics[width, height, ...]{}
\setkeys{Gin}{width=\maxwidth,height=\maxheight,keepaspectratio}
% Set default figure placement to htbp
\makeatletter
\def\fps@figure{htbp}
\makeatother
\setlength{\emergencystretch}{3em} % prevent overfull lines
\providecommand{\tightlist}{%
  \setlength{\itemsep}{0pt}\setlength{\parskip}{0pt}}
\setcounter{secnumdepth}{-\maxdimen} % remove section numbering
\ifluatex
  \usepackage{selnolig}  % disable illegal ligatures
\fi

\title{Yu-Gi-Oh! Meets Data Analysis!}
\author{Gabriel SH (yeye), tutored by Geeked.id}
\date{9/9/2021}

\begin{document}
\maketitle

{
\setcounter{tocdepth}{2}
\tableofcontents
}
\hypertarget{introduction}{%
\section{1. Introduction}\label{introduction}}

\hypertarget{a-bit-about-me}{%
\subsection{1.1 A Bit About Me}\label{a-bit-about-me}}

Hellow, I'm Gabriel, but you may call me Yeye for convenience, and I
would prefer you to call me that :)

\hypertarget{a-bit-about-yu-gi-oh}{%
\subsection{1.2 A bit about Yu-Gi-Oh!}\label{a-bit-about-yu-gi-oh}}

In this portofolio builder, I would like to to analyze
\textbf{Yu-Gi-Oh!} card game, spesifically the TCG (English) version.
YGO is in my kind of \emph{biased} opinion, the best card game I've ever
played, and I stumbled upon this dataset by chance.

Sadly though, I have left playing it since long ago because of
\textbf{money} issues. If you don't know, a single card can cost a
maximum of \textbf{Rp 1.000.000}, or maybe more.

\hypertarget{a-little-note-about-this-portofolio}{%
\subsection{1.3 A Little Note About This
Portofolio}\label{a-little-note-about-this-portofolio}}

In this portofolio, I want to explore \texttt{tidyverse}, which is R
package (more like a bundle), that can \texttt{generalize} the style of
data analysis using R, since many people have different coding styles /
approach, using \texttt{tidyverse} makes the code style uniform and much
more simple and easy to understand.

\begin{Shaded}
\begin{Highlighting}[]
\CommentTok{\#install.packages("tidyverse")}
\FunctionTok{library}\NormalTok{(dplyr)}
\end{Highlighting}
\end{Shaded}

\begin{verbatim}
## 
## Attaching package: 'dplyr'
\end{verbatim}

\begin{verbatim}
## The following objects are masked from 'package:stats':
## 
##     filter, lag
\end{verbatim}

\begin{verbatim}
## The following objects are masked from 'package:base':
## 
##     intersect, setdiff, setequal, union
\end{verbatim}

\hypertarget{meet-the-data}{%
\section{2. Meet The Data}\label{meet-the-data}}

\hypertarget{dataset-brief}{%
\subsection{2.1 Dataset Brief}\label{dataset-brief}}

The datasets are provided by
\href{https://www.kaggle.com/jpalmer2}{James Palmer} from
\texttt{kaggle.com}, and the data itself was scraped from
\href{https://yugioh.fandom.com/wiki/Yu-Gi-Oh!}{Yu-Gi-Oh! Fandom Wiki}.
It consist of 4 files, which upon slight inspection, it's actually quite
similar.

\hypertarget{dataset-characteristics}{%
\subsection{2.2 Dataset Characteristics}\label{dataset-characteristics}}

\hypertarget{parameters}{%
\subsubsection{2.2.1 Parameters}\label{parameters}}

\begin{itemize}
\tightlist
\item
  \textbf{Name} : Card name
\item
  \textbf{Card.Type} : Monster / Spell / Trap card
\item
  \textbf{Attribute} : Monster Attribute (LIGHT, DARK, EARTH, WATER,
  etc)
\item
  \textbf{Monster.Type} : Monster type (Fairy, Zombie, Fiend, Pyro,
  etc), also represents the Monster Type (Synchro, Effect, Xyz, etc)
\item
  \textbf{Level.Rank} : Monster Level / Xyz Monster Rank
\item
  \textbf{ATK.DEF} : ATK \& DEF of the monster
\item
  \textbf{Passcode} : Card code printed on the bottom left side
\item
  \textbf{Materials\ldots Ritual.spell} : Material for Synchro Monsters
  / Fusion Monsters / Xyz Monsters / Link Monsters
\item
  \textbf{Effect.Type} : This is a bit complex, but I can give you an
  easy example, suppose it's \emph{Continuous} , then as long as that
  card is face up on the field, the effect will remain.
\item
  \textbf{Effect} : The actual effect of the card printed on it.
\item
  \textbf{Spell.Trap.type} : The type of the Spell/Trap card, like
  \emph{Equip} / \emph{Continuous} / \emph{Ritual} / etc
\item
  \textbf{TCG.sets} : This represents the package code where the card
  originated from. A package can be a \emph{Structure Deck} ,
  \emph{Collector Tin} , \emph{Booster Pack} , etc. A card may have more
  than 1 package code because there's a possibility of ``Reprinting''
  the card to introduce lower money-cost
\item
  \textbf{Ban\_list} : Banlist status of the card, either
  \emph{Unlimited} / \emph{Semi Limited} / \emph{Limited} /
  \emph{Forbidden}
\item
  \textbf{Number.of.sets} : If it's more than 1, then the card has
  already been reprinted.
\item
  \textbf{Link.Arrows} : The direction of \emph{Link Arrow} for Link
  Monster, it's octagonal (8-way) and may consist of more than 1 arrow.
  The number of arrows also represents the \textbf{Link Rating}, which
  is like Level / Rank for other Monster.
\item
  \textbf{Pendulum.Scale} : The pendulum scale of Pendulum Monsters.
\item
  \textbf{Set.Name} : The card package name
\item
  \textbf{Relase.Date} : Well\ldots{} the release date.
\end{itemize}

\begin{Shaded}
\begin{Highlighting}[]
\FunctionTok{names}\NormalTok{(yugi)}
\end{Highlighting}
\end{Shaded}

\begin{verbatim}
##  [1] "X"                        "Name"                    
##  [3] "Card.type"                "Attribute"               
##  [5] "Monster.Type"             "Level.Rank"              
##  [7] "ATK.DEF"                  "Passcode"                
##  [9] "Materials...Ritual.spell" "Effect.type"             
## [11] "Effect"                   "Spell.Trap.type"         
## [13] "TCG.sets"                 "Ban_list"                
## [15] "Number.of.sets"           "Link.Arrows"             
## [17] "Pendulum.Scale"           "Rarities"                
## [19] "Set.Name"                 "Release.Date"
\end{verbatim}

\hypertarget{a-little-peek-to-the-data}{%
\subsubsection{2.2.2 A Little Peek to the
Data}\label{a-little-peek-to-the-data}}

\begin{Shaded}
\begin{Highlighting}[]
\FunctionTok{str}\NormalTok{(yugi)}
\end{Highlighting}
\end{Shaded}

\begin{verbatim}
## 'data.frame':    22293 obs. of  20 variables:
##  $ X                       : int  0 1 2 3 4 5 6 7 8 9 ...
##  $ Name                    : chr  "\"A\" Cell Breeding Device" "\"A\" Cell Incubator" "\"A\" Cell Recombination Device" "\"A\" Cell Scatter Burst" ...
##  $ Card.type               : chr  "Spell " "Spell " "Spell " "Spell " ...
##  $ Attribute               : chr  "" "" "" "" ...
##  $ Monster.Type            : chr  "" "" "" "" ...
##  $ Level.Rank              : num  NA NA NA NA NA NA NA NA NA NA ...
##  $ ATK.DEF                 : chr  "" "" "" "" ...
##  $ Passcode                : chr  "34541863" "64163367" "91231901" "73262676" ...
##  $ Materials...Ritual.spell: chr  "" "" "" "" ...
##  $ Effect.type             : chr  "Trigger-like" "Continuous-like, Trigger-like" "Effect, Ignition-like" "Effect" ...
##  $ Effect                  : chr  "During each of your Standby Phases, put 1 A- Counter on 1 face-up monster your opponent controls" "Each time an A- Counter(s) is removed from play by a card effect, place 1 A- Counter on this card. When this ca"| __truncated__ "Target 1 face-up monster on the field; send 1 \" Alien\" monster from your Deck to the Graveyard, and if you do"| __truncated__ "Select 1 face-up \" Alien\" monster you control. Destroy it and distribute new A- Counters equal to its Level a"| __truncated__ ...
##  $ Spell.Trap.type         : chr  "Continuous " "Continuous " "Quick-Play " "Quick-Play " ...
##  $ TCG.sets                : chr  "FOTB-EN043" "GLAS-EN062" "INOV-EN063" "STON-EN041" ...
##  $ Ban_list                : chr  "Unlimited " "Unlimited " "Unlimited " "Unlimited " ...
##  $ Number.of.sets          : int  1 1 1 1 1 1 3 3 3 1 ...
##  $ Link.Arrows             : chr  "" "" "" "" ...
##  $ Pendulum.Scale          : num  NA NA NA NA NA NA NA NA NA NA ...
##  $ Rarities                : chr  "Common" "Common" "Common" "Common" ...
##  $ Set.Name                : chr  "Force of the Breaker Sneak Peek Participation Card" "Gladiator's Assault Sneak Peek Participation Card" "Invasion: Vengeance Sneak Peek Participation Card" "Strike of Neos Sneak Peek Participation Card" ...
##  $ Release.Date            : chr  "May 16, 2007" "November 14, 2007" "November 4, 2016" "February 28, 2007" ...
\end{verbatim}

\hypertarget{problem}{%
\subsection{2.3 Problem}\label{problem}}

Firstly, of course I will need the name of the card.

\hypertarget{processing-the-data}{%
\section{3. Processing the Data}\label{processing-the-data}}

\hypertarget{data-wrangling}{%
\subsection{3.1 Data Wrangling}\label{data-wrangling}}

\hypertarget{data-visualization}{%
\subsection{3.2 Data Visualization}\label{data-visualization}}

\hypertarget{infering-the-data}{%
\section{4. Infering the Data}\label{infering-the-data}}

\hypertarget{what-we-have-got-so-far}{%
\subsection{4.1 What We Have Got So Far}\label{what-we-have-got-so-far}}

\hypertarget{conclusion}{%
\subsection{4.2 Conclusion}\label{conclusion}}

\end{document}
